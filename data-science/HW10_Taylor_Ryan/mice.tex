\documentclass[11pt,english]{article}

%%%%%%%%%%%%%%%%%%%%%%%%%%%%%%%%%%%%%%%%%%%%%%%%%%%%%%%%%%%
% Packages
%%%%%%%%%%%%%%%%%%%%%%%%%%%%%%%%%%%%%%%%%%%%%%%%%%%%%%%%%%%

\usepackage{fullpage}
\usepackage[showframe=false,margin=1in]{geometry}
\usepackage{amsmath, amsthm, amssymb}
\usepackage[linesnumbered,ruled,vlined]{algorithm2e}
\usepackage{url}
\usepackage{fancyvrb}

\usepackage{graphicx}
\usepackage{listings}

\parindent=0pt
\setlength{\parskip}{1em}
\lstset{
  basicstyle=\fontsize{10}{13}\selectfont\ttfamily
}


\begin{document}

\title{CSC 4780 \\
Fall 2022\\ Homework 10 - Mice Analysis}
\author{Ryan Taylor \\ rtaylor80@student.gsu.edu}
\maketitle

\section{Contingency Table}

\begin{tabular}{l|rr|r}
 Gene & No Cancer & Has Cancer & Total \\
\hline
 J   &  93 & 37 & 130 \\
 K   &  34 &  5 &  39 \\
 R   &  20 &  1 &  21 \\
 \hline
     & 147 & 43 & 190 \\
\end{tabular}

\section{Conditional Proportions}

\begin{tabular}{l|rr|r}
 Gene & No Cancer & Has Cancer & Total \\
\hline
 J   & 71.54\% & 28.46\% & 130  \\
 K   & 87.18\% & 12.82\% &  39  \\
 R   & 95.24\% &  4.76\% &  21  \\
 \hline
     & 77.37\% & 22.63\% & 190  \\
\end{tabular}

\section{Expected counts if the gene and cancer were independent}

\begin{tabular}{l|rr|r}
 Gene & No Cancer & Has Cancer & Total \\
\hline
 J   & 100.6 & 29.4 & 130 \\
 K   & 30.17  &  8.8 &  39 \\
 R   & 16.3  &  4.8 &  21 \\
 \hline
     & 147    & 43    & 190 \\
\end{tabular}

\section{Chi-Squared}
$X^2$ = 8.497206532327557

\section{Degrees of Freedom}
Degrees of Freedom = 2

\section{P-Value}
p = 0.014284171167428195

\section{Conclusion}
Since the p-value is less than 0.05, it can be concluded that the gene and cancer are likely not independent.

\end{document}
